\documentclass{mall}

\newcommand{\version}{Version 1.0}
\author{Elliot Johansson, \url{elljo130@student.liu.se}\\
  Nadim Lakrouz, \url{nadla777@student.liu.se}}
\title{Gruppkontrakt}
\date{2022-09-14}
\rhead{}


\begin{document}
\projectpage

\section{Förutsättningar}
\label{prereq}



\begin{itemize}
\item \textbf{Följande saker vill jag att min/mina kollegor visar hänsyn och förståelse för}

Online kommunikation.

\item \textbf{Hur ska jag bete mig för att stötta min/mina kollegor utifrån sina förutsättningar?}

Se till att alla förstår innehållet.

\end{itemize}

\section{Hur vi arbetar tillsammans}

\begin{itemize}
\item \textbf{Vilka tider arbetar vi, och vilka tider är vi nåbara utöver detta?}

På utsatta labbtider och  lediga arbetstider där båda är tillgängliga.

\item \textbf{Hur kommunicerar vi med varandra? Vilka verktyg/kanaler använder vi? Hur och när är det okej att vi avbryter varandra?}

Discord och snapchat. mellan 10:00 och 00:00 men det kan vara så att vi sover så då får man vänta tills nästa dag.

\item \textbf{Hur gör vi för att ge varandra möjlighet att framföra åsikter och tankar om uppgifter och idéer till arbetet?}

Vi skapar en säker och öppen arbetsmiljö där alla är bekvämma med att yttra åsikter och tankar.

\item \textbf{Hur ofta tar vi paus? Ska vi hjälpas åt att påminna varandra om att ta paus?}

På labbtider så är det inte så nödvändigt att ta en paus. Men om vi skulle jobba längre sessioner så kan vi försöka ta en paus varje timme för att hålla humöret på topp. Om nån av oss känner att vi behöver en paus så är det bara att säga till.

\item \textbf{Arbetar vi tillsammans med uppgifter, eller var för sig?}

I början arbetar vi tillsammans men om vi känner att arbetet tjänar på att dela upp arbetet kan det hända att vi gör det också.

\item \textbf{Hur bestämmer vi vem som gör vad?}

Genom att diskutera.

\item \textbf{Hur specifierar vi vad som ingår i varje uppgift, och när den är klar?}

Genom en to-do lista kan vi dela upp uppgiften och se tydligt vilka moment som ska ingå. När vi är nöjda med resultatet.

\item \textbf{Hur snabbt förväntar vi oss att en uppgift kan vara klar?}

Beroende på storlek av uppgift kan det variera men ett par dagar innan mjuk deadline är optimalt och ger tillfälle för att komplettera.

\item \textbf{Hur håller vi reda på att uppgifter vi identifierat inte glöms bort?}

Genom en väldigt fin to-do list.

\end{itemize}

\section{Om jag tycker att något inte fungerar}


\begin{itemize}
\item \textbf{Vad gör vi om någon kommer sent?}

Om någon kommer sent så får de se till att meddela det. Man får respektera att det kan hända saker enstaka gånger som gör att man inte kan meddela men om någon kommer sent upprepat många gånger får man ta att diskutera arbetstider och arbetsbelastning och se om det går att göra kompromisser.

\item \textbf{Vad gör vi om någon inte slutför sina uppgifter?}

Hjälper till.

\item \textbf{Vad gör vi om arbetsfördelningen blir ojämn?}

Försöker att lösa det med to-do list och bra kommunikation.

\item \textbf{Hur tar vi upp ett problem med berörda personer?}

Vi pratar om det och om någon inte är bekväm att tala med den andra personen får man plocka in en tredje part som medlar mellan personerna.

\item \textbf{Hur ger jag kritik och beröm till andra personer i gruppen?}

Se till att ge konstruktiv kritik på ett vänligt sätt. Se till att ge beröm när någon gör något bra.

\end{itemize}

\section{Utvärdering}


\begin{itemize}
\item \textbf{När ska vi påminna oss om gruppkontraktet och utvärdera hur det fungerat?}

Om det märks att arbetet inte fungerar kan vi gå tillbaka till kontraktet och gå igenom det.

\end{itemize}

\end{document}
