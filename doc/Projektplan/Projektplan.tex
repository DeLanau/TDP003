\documentclass{TDP003mall}



\newcommand{\version}{Version 1.1}
\author{Elliot Johansson, \url{elljo130@student.liu.se}\\
  Nadim Lakrouz, \url{nadla777@student.liu.se}}
\title{Projektplan}
\date{2022-09-29}
\rhead{Elliot Johansson\\
Nadim Lakrouz}



\begin{document}
\projectpage
\section{Revisionshistorik}
\begin{table}[!h]
\begin{tabularx}{\linewidth}{|l|X|l|}
\hline
Ver. & Revisionsbeskrivning & Datum \\\hline
0.1 & Första utkastet &  09/19\\\hline
0.2 & andra versionen & 09/29 \\\hline
\end{tabularx}
\end{table}


\section{Introduktion}
I detta projekt så ska en portfolio skapas. Den ska användas för att ladda upp projekt och i framtiden möjligtvis använda detta för att visa vad man har gjort tidigare. Projektet är uppdelat i två delar: Datalagret och Presentationslagret. De i sin tur består av json, html, css och andra API. 



\subsection{Metoder}
tidsplanering kommer följas med anpassningar beroende prioriteringar av arbetet som måsta vara klart på specifik vecka. Arbetet kommer ske delvis på plats och delvis på discord. För att vi alltid ska använda senaste versionen av projektet kommer git att användas.

\subsection{Tekniker}

Datalagret - Python3, json \\
Presentationslagret - HTML, css, flask, jinja

\section{Planering}

\subsection{Milstolpar}

Onsdag 09/28 Implementera search funktion i datalagret
\\
Måndag 10/03 Implementera dataöverföring från datalagret till presentationslagret
\\
Onsdag 10/05 Fungerande search bar och tag search funktion på websidan

\section{Vecko planering}

Prio 1 - Det måste vara klart den här veckan
\\
Prio 2 - Det är bra om det fixas men inte ett måste
\\
Prio 3 - Om extra tid finns så kan detta fixas

\subsection{Vecka 37}
\begin{table}[!h]
  \begin{tabularx}{\linewidth}{|l|X|r|r|r|}
\hline

Aktivitet & beskrivning & prio & estimerad tid & faktisk tid \\\hline
tidsplan & göra klart tidsplan & 1 & 2t & 2t\\\hline
LoFi prototyp & göra klart LoFi prototyp & 1 & 2t & 3t \\\hline

\end{tabularx}
\end{table}

\subsection{Vecka 38}

\begin{table}[!h]
  \begin{tabularx}{\linewidth}{|l|X|r|r|r|}
\hline

Aktivitet & beskrivning & prio & estimerad tid & faktisk tid \\\hline
Projektplan & första utkast & 1 & 2t & 3t \\\hline

\end{tabularx}
\end{table}

\subsection{Vecka 39}

\begin{table}[!h]
  \begin{tabularx}{\linewidth}{|l|X|r|r|r|}
\hline

Aktivitet & beskrivning & prio & estimerad tid & faktisk tid \\\hline
datalagret klart & göra klart datalagret & 1 & 6t & 10t \\\hline

\end{tabularx}
\end{table}

\subsection{Vecka 40 	}

\begin{table}[!h]
  \begin{tabularx}{\linewidth}{|l|X|r|r|r|}
\hline

Aktivitet & beskrivning & prio & estimerad tid & faktisk tid \\\hline
Presentationslagret & Skapa presentationslagret och koppla ihop det med datalagret & 1 & 6t & -\\\hline
Presentationslagret utseende & Se till att sidan ser snygg ut & 2 & 6t & -\\\hline

\end{tabularx}
\end{table}

\subsection{Vecka 41}

\begin{table}[!h]
  \begin{tabularx}{\linewidth}{|l|X|r|r|r|}
\hline

Aktivitet & beskrivning & prio & estimerad tid & faktisk tid \\\hline
presentationslagret & vara till största delen klar med presentationslagret & 1 & 3t & -\\\hline
systemdemonstration & demonstrera systemet för andra grupper & 1 & 2t & 2t\\\hline
publicera portfolio & skicka in portfolion i dess helhet & 1 & 1t & -\\\hline
första utkast av systemdokumentationen & skicka in första versionen av systemdokumentationen & 1 & 2t & -\\\hline

\end{tabularx}
\end{table}

\subsection{Vecka 42}

\begin{table}[!h]
  \begin{tabularx}{\linewidth}{|l|X|r|r|r|}
\hline

Aktivitet & beskrivning & prio & estimerad tid & faktisk tid \\\hline
testdokumentation & lämna in & 1 & 2t & -\\\hline
individuell reflektionsdokument & lämna in & 1 & 2t & -\\\hline
systemdokumentation inlämning & fixa de sista bristerna i systemdokumentationen & 1 & 2t & -\\\hline

\end{tabularx}
\end{table}

\section{Risker}
Eventuella risker som kan förekomma är att någon av oss blir sjuk eller att något tar längre tid än förväntat. För att förebygga att sådana händelser påverkar arbetet kan marginaler inkluderas i tidsplanen

\end{document}

